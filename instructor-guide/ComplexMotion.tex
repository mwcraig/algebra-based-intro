\documentclass[fleqn,letterpaper]{article}
\usepackage{fullpage}
\usepackage[dvips]{graphicx}
\usepackage{amssymb}
\usepackage{fancyhdr}
\usepackage[active]{srcltx}
\addtolength{\parskip}{\baselineskip}
\pagestyle{fancy}
\headheight=12pt
\parindent 0cm

\begin{document}

\lhead{\it Instructor's Lab Manual for Physics 160 }
\cfoot{}
\rhead{\it Page \thepage~of \pageref{LastPage} }
\headsep=25pt
%\baselineskip=12pt

\section*{Complex Motion}

\subsection*{Additional Equipment}

\begin{itemize}
  \item{The additional items listed should have been removed, the students will only use the carts and video.  They could do an A-level with the fan carts, though.}
\end{itemize}

\subsection*{Objective}

This week's lab builds off of the previous week's lab, ``Simple Motion''.  In ``Simple Motion'' the students gained experience with
%
\begin{itemize}
 \item{plots of position/velocity/acceleration}
 \item{the slope of position vs time giving the velocity}
 \item{the slope of velocity vs time giving the acceleration}
 \item{the area under velocity vs time giving the change in position}
 \item{the area under acceleration vs time giving the change in velocity}
\end{itemize}
%
This week's lab, ``Complex Motion'', will extend these ideas to 2-D (projectile) motion.  We have covered all of the individual concepts in lecture, but this will be the first time the students tackle a problem in 2 dimensions.


\subsection*{Conceptual (C-level) (Done BEFORE Lab)}

\begin{itemize}
 \item{The students should have 6 separate plots, 3 for each direction.  They have read about free-fall, and we have done examples in class getting a velocity plot from a ``curvy'' position plot, as well as an acceleration plot from the velocity plot.}
 \item{Students should state that the motion of the object shouldn't change, even if its mass does.  The explanation hinges on understanding/knowing that the free-fall acceleration is always $a_y = - g$, regardless of the object's mass.}
 \item{The students should see this reflected in the simulation, as long as they do not have the ``Air Resistance'' box checked!}
 \item{The velocity vector gets longer when it points in the same direction as the acceleration vector (so the object speeds up).}
 \item{The velocity vector gets shorter when it points in the opposite direction of the acceleration vector (so the object slows down).}
 \item{The velocity vector changes direction (but remains the same size) when the acceleration vector is perpendicular to the velocity vector.}
\end{itemize}

\subsubsection*{Rubric}

\begin{itemize}
 \item{Pick a notebook at random from the group.  If the entire C-Level has not been attempted, dock 1 point from the group.}
 \item{If the students just ``quit'' or otherwise have a bad attitude are are not staying on task, dock a point (or two) after a warning.}
 \item{Other than these 2 things, assist the groups by asking leading questions as you see fit so they understand the C-level.  If they came prepared, worked hard, and got through the C-level, they get 4/4.}
\end{itemize}


\subsection*{Basic Lab (B-level)}

The B-level gives the students practice interpreting real data for projectile motion, to (hopefully) verify their C-Level answers.  As in the previous lab, the students need to think about he the region of interest (ROI) of the plots.  A critical (and not trivial) skill that the students need to learn is how to pick out the relevant data from the plots.

\begin{itemize}
\item{The students can insert the video using Insert $\rightarrow$ Movie, then going to the ``Tutorials'' Folder.}
\item{To get acceleration plots, the students will need to add ``Calculated Columns'' to compute the acceleration.  They should use the derivative of the velocity columns.  Since these are 160 students (and not expected to know what a derivative is), these instructions will be written on the board.}
\item{When analyzing the data (picking the ROI), the students will need to be careful to look at a region after the ball has been pushed/thrown, and before the ball hits the ground.}
\item{The horizontal position should have constant slope, and the horizontal velocity should be flat.}
\item{The vertical position should be a parabola, the veritcal velocity should have a constant (negative) slope, and the vertical acceleration should be flat and negative (hopefully at $\sim-10~{\rm m/s^2}$.}
\item{When extracting numbers from the plots, it is crucial that the students establish an appropriate scale (there is a \textbf{2-meter} stick in the video on the ground).  There is a tool in the video window that will let them set a scale.}
\item{Students should measure the slope of horizontal position vs time (for the ROI only) using ``Curve Fit'', and compare this with the average value of the velocity plot over the same ROI.}
\item{Repeat the above for comparing the acceleration and velocity plots for the vertical motion.}
\item{For ``something vs time'' plots in LoggerPro, students do not need to add error bars to the data points generated by the sensors/probes.  However, we do want them to recognize the uncertainty in the fitted values.  This ties in with the uncertainties in the model from last week's lab.}
\end{itemize}

\subsubsection*{Rubric -- Lab Summary}

This rubric applies to the lab summary the students will hand in at the beginning of the following lab (one per lab group).  Page 3 has a description of what a good lab summary contains, and there are generic rubrics found in the front of the lab manual.  

This lab is one of the two early ``graph-heavy'' labs with kinematics.  Future lab reports will have fewer plots the students need to turn in and analyze.

For this lab, specifically, they should have:

\begin{enumerate}
 \item{
  \begin{itemize}
   \item{Some plots and numbers present, very minimal.}
  \end{itemize}
}
 \item{
  \begin{itemize}
   \item{All 3 plots (each with horizontal and vertical data) present, described, labeled correctly (including units)}
   \item{Some fit/averages present on the plots}
   \item{Some uncertainties missing}
   \item{Reported results, but no interpretation}
   \item{Missing correct scale on plots (didn't convert from pixels to meters)}
  \end{itemize}
}
 \item{
  \begin{itemize}
   \item{All uncertainties present}
   \item{All fits/averages present on the plots}
   \item{Some of the slope/avg comparisons done.}
   \item{Some physics, but no explicit mention of the models used to explain each plot: $\Delta x = v_x \Delta t$, $\Delta y = v_{0,y} \Delta t + \frac{1}{2}a_y \Delta t^2$, or $v_{f,y} = v_{i,y} + a_y \Delta t$.}
  \end{itemize}
}
 \item{
  \begin{itemize}
   \item{They should state that they expect there to be a linear relationship between position and time (at constant $v$) on the horizontal axis, given by $\Delta x = v_x \Delta t$}
   \item{They should state that they expect there to be a quadratic relationship between position and time (at constant $a$) on the vertical axis, given by $\Delta y = v_{i,y} \Delta t + \frac{1}{2}a_y \Delta t^2$}
   \item{They should state taht they expect there to be a linear relationship between velocity and time (at constant $a$) on the vertical axis, given by $\Delta v_y = a_y \Delta t$.}
   \item{They correctly compare (using uncertainties):
    \begin{itemize}
    \item{the slope of $x$ vs $t$ with the average value of $v_x$}
    \item{the slope of $v_y$ vs $t$ with the average value of $a_y$ vs $t$}
    \end{itemize}}
  \end{itemize}
}
\end{enumerate}


\subsection*{Advanced/Extended Lab Ideas (A-level)}

Students may pick a single A-level to do.  They do not need to stay with their in-class group (though many choose to).  Some of these are identical to the previous week's labs, students should be made aware that they cannot REPEAT the same A-Level from last week.

\begin{itemize}
 \item{The first suggestion asks them to investigate how mass affects the acceleration on an incline.  They \textbf{should} find that mass has no effect, though there could be some friction present that would cause their results to not be ``ideal''.}
 \item{The third and fourth suggestions ask them to investigate how the ``area under the curve'' gives a ``change in'' value from a different plot.  The question about units is meant to guide them towards this.  For example, the area under the acceleration vs time curve has units of ${\rm m/s^2} \times {\rm s} = \rm{m/s}$, the units of velocity.  Students should pick 5 different ROIs in which to make their comparisons.  For example, if the ROI is from 1s to 4s, they could choose 1s-4s, 1.5s-3s, 2.0s-4s, etc.}
\end{itemize}

\subsubsection*{Rubric -- Lab Summary}

Follow the rubric on page 7 of the lab manual.  Since the students can choose any A-level, I won't be writing up a detailed rubric for every suggestion.  Hopefully, the B-level rubrics included here help to establish a good guideline.  If you have further questions, please ask!

\label{LastPage}

\end{document}